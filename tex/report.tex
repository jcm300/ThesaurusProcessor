\documentclass{llncs}
\usepackage[utf8]{inputenc}
\usepackage{fancyvrb} 
\usepackage[portuguese]{babel}
\usepackage{ragged2e}

\begin{document} \mainmatter
\title{Processador de Thesaurus}
\titlerunning{Processador de Thesaurus}
\author{José Carlos Lima Martins A78821 \and
        Miguel Miranda Quaresma A77049}
\authorrunning{José Carlos Lima Martins A78821 \and
        Miguel Miranda Quaresma A77049}
\institute{                                                                
University of Minho, Department of  Informatics, Braga, Portugal\\
e-mail: \{a78821,a77049\}@alunos.uminho.pt
}

\maketitle

\justify

\begin{abstract}
Este documento serve de apoio ao desenvolvimento de um Processador Thesaurus explicando todas as decisões tomadas de modo a obter o mesmo. Inicialmente será explicada a estrutura dos ficheiros de entrada (extensão .mdic) bem como mostradas as decisões tomadas de modo a obter o produto final. Por fim, serão apresentados exemplos de utilização do mesmo.
\end{abstract}

\section{Introdução}
O presente projeto (Processador de Thesaurus) visa ser um sistema de processamento de dicionários Thesaurus que recorre a expressões regulares (ERs) para filtrar e transformar os mesmos, extraindo e tratando a informação mais relevante de forma eficiente. Para atingir este objetivo é usada a linguagem de filtragem e tratamento de dados \texttt{AWK} visto ser uma \textbf{DSL}(Domain Specific Language) com foco em dados semi-estruturados.

\section{Preliminares}
De modo a compreender melhor o desenvolvimento deste projeto é importante saber e compreender a estrutura dos dicionários (ficheiros .mdic). 
A estrutura de um dicionário pode ser dividido em três conceitos:
\begin{itemize}
    \item linhas começadas por '\#': comentários a ignorar
    \item diretivas gerais:
        \begin{itemize}
            \item linhas começadas por '\%dom: dominio': indica que até ao aparecimento de nova linha começada por '\%dom:' todos os termos são de 'dominio', sendo dom uma relação e a sua inversa é voc
            \item linhas começadas por '\%inv: relação1 : relação2': indica que a relação1 tem como inversa a relação2
        \end{itemize}
    \item linhas começadas por '\%THE': indicam tabelas de relações, com as seguintes caracteristicas:
        \begin{itemize}
            \item a linha inicial possui as relações bem como as classes da coluna correspondente
            \item cada linha tem 1 ou mais termos, menos a inicial que possui apenas relações e classes
            \item os termos são separados por ':' daqueles com que se relacionam
            \item a relação entre o termo da coluna 1 e da coluna N é dado pela relação N da linha inicial
            \item na presença de vários termos com a mesma relação, podem ser agrupados por '$|$'
            \item quando uma relação na linha inicial possui $<$classe, ou seja 'relação$<$classe', significa que o elementos dessa coluna são instancia da classe
            \item quando '\%THE$<$classe' significa que o termo 1 é instancia da classe
        \end{itemize}
\end{itemize}
Pode-se finalmente, passar ao desenvolvimento do processador Thesaurus.

\section{Desenvolvimento}
\subsection{Exe 1}
O processamento iniciou-se pela identificação dos domínios e relações presentes no dicionário. Para tal, e tendo em conta a estrutura dos documentos, foi usado como \texttt{Field Separator} o carácter ':'.
Tendo a conta a sintaxe de Padrão $\to$ Ação inerente ao \texttt{AWK} foram considerados relevantes os seguintes padrões:
\begin{enumerate}
    \item \verb|/^%dom/|: padrão que identifica novo domínio no início de uma linha
    \item \verb|/^%THE/|: padrão que identifica início de tabela de relações indicadas na presente linha
\end{enumerate}
No caso do padrão \verb|/^%dom/| a ação compreende a remoção de espaços que precedam o domínio e o armazenamento do mesmo numa matriz \texttt{ind}.
\begin{Verbatim}
    /^%dom/     {sub(/^ /,"",$2); ind[0][$2]++}
\end{Verbatim}
Quando é encontrado o padrão \verb|/^%THE/| são percorridos todos os campos (\textbf{i.e.}relações) e quando um dada relação não vazia (\verb|$j!=" " && $j!=""|) é encontrada são-lhe removidos os espaços precedentes e subsequentes removendo também, eventualmente, a indicação de classe. De seguinta estaes são também armazenados na matriz \texttt{ind}.
\begin{Verbatim}
/^%THE/     {for(j=2;j<=NF;j++){
                if($j != " " && $j!=""){
                    sub(/^ /,"",$j);           
                    sub(/(\s)|(<(.*)?)/,"",$j);
                    !ind[1][$j]++;
                }                              
             }
            }

\end{Verbatim}

Por fim, quando o final de ficheiro é encontrado (\texttt{EOF}) são impressos os domínios e as relações recorrendo, para isso, às funções \texttt{printDominios} e \texttt{printRelations} respetivamente.

\subsubsection{Usage}

\section{Exe 2}
O objetivo deste exercício é mostrar os triplos expandidos(um por cada linha) correspondentes. Como tal, há a necessidade de percorrer os ficheiros por completo. Consideramos como Field Separator ':'. Consoante o início da linha, o processamento é indicado de seguida:
\begin{itemize}
    \item \verb|/^%dom/|: guarda-se o dominio indicado numa variável até aparecer uma nova linha com o mesmo padrão altura em que o dominio é atualizado
    \item \verb|/^%inv/|: guarda num índice \texttt{inv} a relação e a sua respetiva inversa
    \item \verb|/^%THE/|: as relações são armazenadas num índice \texttt{relation} e as classes (quando aplicável/existentes) num índice \texttt{class}
    \item \verb|$1 !~/^%/ && $0 !~ /^#/ && $0 != ""| (\textbf{i.e.}tabelas de relações): os triplos são armazenados no índice \texttt{triples} da seguinte maneira e consoante as seguintes condições: 
        \begin{itemize}
            \item termo1 relacionado com o dominio(\textbf{dom}) bem como a respetiva inversa(\textbf{voc})
            \item caso exista uma classe associada ao '\%THE' é adicionado os triplos correspondentes ao \textbf{instance of}(iof) e \textbf{tem como instancia}(inst) envolvendo o termo1 e a classe
            \item são percorridos o resto dos campos (termos da lado direito), separando-os por $|$ usando o \texttt{split} obtendo as seguintes triplos dependendo do caso:
                \begin{itemize}
                    \item os triplos que relacionam o termo com o dominio (dom e voc)
                    \item o triplo envolvendo o termo1, a relação da coluna correspondente e o termo, sendo que caso a relação possua inversa(verificado no indice inv) é adicionado o triplo
                    \item caso a coluna possua classe, o termo é relacionado com a mesma(iof e inst)
                \end{itemize}
        \end{itemize}
\end{itemize}
Após a adição de todos os triplos no indice triples, ou seja após percorridos todos os ficheiros(END) é percorrido esse mesmo indice imprimindo cada um. Este indice possui como elemento diferenciador o triplo e como conteudo se ocorreu, de modo a retirar triplos repetidos. É também bom referir que todos os elementos, seja classes, relações ou termos é lhes retirado espaços que estejam a mais antes e após o mesmo.

\subsubsection{Usage}
O exemplo mais dificil de utilização deste programa é:
\begin{verbatim}
gawk -f exe2.awk *.mdic
\end{verbatim}
Contudo pode ser também interessante guardar o resultado num ficheiro:
\begin{verbatim}
gawk -f exe2.awk *.mdic >> file.txt
\end{verbatim}

\section{Conclusão}

\end{document}
