\documentclass{llncs}
\usepackage[utf8]{inputenc}
\usepackage{fancyvrb}
\usepackage[portuguese]{babel}
\usepackage{ragged2e}

\begin{document}
\mainmatter
\title{Processador de Thesaurus}
\titlerunning{Processador de Thesaurus}
\author{José Carlos Lima Martins A78821 \and
        Miguel Miranda Quaresma A77049}
\authorrunning{José Carlos Lima Martins A78821 \and
        Miguel Miranda Quaresma A77049}
\institute{                                                                
University of Minho, Department of  Informatics, Braga, Portugal\\
e-mail: \{a78821,a77049\}@alunos.uminho.pt
}

\maketitle

\justify

\begin{abstract}
Este documento serve de apoio ao desenvolvimento de um processador Thesaurus explicando todas as decisões tomadas de modo a obter o mesmo. Inicialmente será explicado a estrutura dos ficheiros de entrada (extensão .mdic) bem como será mostrado as decisões tomadas de modo a chegar ao produto final. Por fim, será apresentado exemplos de utilização do mesmo.
\end{abstract}

\section{Introdução}

\section{Desenvolvimento}

\section{Conclusão}

\end{document}
