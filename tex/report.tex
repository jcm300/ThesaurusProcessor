\documentclass{llncs}
\usepackage[utf8]{inputenc}
\usepackage{fancyvrb}
\usepackage[portuguese]{babel}
\usepackage{ragged2e}

\begin{document}
\mainmatter
\title{Processador de Thesaurus}
\titlerunning{Processador de Thesaurus}
\author{José Carlos Lima Martins A78821 \and
        Miguel Miranda Quaresma A77049}
\authorrunning{José Carlos Lima Martins A78821 \and
        Miguel Miranda Quaresma A77049}
\institute{                                                                
University of Minho, Department of  Informatics, Braga, Portugal\\
e-mail: \{a78821,a77049\}@alunos.uminho.pt
}

\maketitle

\justify

\begin{abstract}
Este documento serve de apoio ao desenvolvimento de um Processador Thesaurus explicando todas as decisões tomadas de modo a obter o mesmo. Inicialmente será explicada a estrutura dos ficheiros de entrada (extensão .mdic) bem como mostradas as decisões tomadas de modo a obter o produto final. Por fim, serão apresentados exemplos de utilização do mesmo.
\end{abstract}

\section{Introdução}
O presente projeto (Processador de Thesaurus) visa ser um sistema de processamento de dicionários Thesaurus que recorre a expressões regulares (ERs) para filtrar e transformar os mesmos, extraindo e tratando a informação mais relevante de forma eficiente. Para atingir este objetivo é usada a linguagem de filtragem e tratamento de dados \texttt{AWK} visto ser uma \textbf{DSL}(Domain Specific Language) com foco em dados semi-estruturados.

\section{Preliminares}
De modo a compreender melhor o desenvolvimento deste projeto é importante saber e compreender a estrutura dos dicionários (ficheiros .mdic). 
A estrutura de um dicionário pode ser dividido em três conceitos:
\begin{itemize}
    \item linhas começadas por '\#': comentários a ignorar
    \item diretivas gerais:
        \begin{itemize}
            \item linhas começadas por '\%dom: dominio': indica que até ao aparecimento de nova linha começada por '\%dom:' todos os termos são de 'dominio', sendo dom uma relação e a sua inversa é voc
            \item linhas começadas por '\%inv: relação1 : relação2': indica que a relação1 tem como inversa a relação2
        \end{itemize}
    \item linhas começadas por '\%THE': indicam tabelas de relações, com as seguintes caracteristicas:
        \begin{itemize}
            \item a linha inicial possui as relações bem como as classes da coluna correspondente
            \item cada linha tem 1 ou mais termos, menos a inicial que possui apenas relações e classes
            \item os termos são separados por ':' daqueles com que se relacionam
            \item a relação entre o termo da coluna 1 e da coluna N é dado pela relação N da linha inicial
            \item na presença de vários termos com a mesma relação, podem ser agrupados por '$|$'
            \item quando uma relação na linha inicial possui $<$classe, ou seja 'relação$<$classe', significa que o elementos dessa coluna são instancia da classe
            \item quando '\%THE$<$classe' significa que o termo 1 é instancia da classe
        \end{itemize}
\end{itemize}
Pode-se finalmente, passar ao desenvolvimento do processador Thesaurus.

\section{Desenvolvimento}

\section{Exe 2}
O objetivo deste exercício é mostrar os triplos expandidos(um por cada linha) correspondentes. Como tal, há a necessidade de percorrer os ficheiros por completo 

\section{Conclusão}

\end{document}
