\documentclass{llncs}
\usepackage[utf8]{inputenc}
\usepackage{fancyvrb}
\usepackage[portuguese]{babel}
\usepackage{ragged2e}

\begin{document}
\mainmatter
\title{Processador de Thesaurus}
\titlerunning{Processador de Thesaurus}
\author{José Carlos Lima Martins A78821 \and
        Miguel Miranda Quaresma A77049}
\authorrunning{José Carlos Lima Martins A78821 \and
        Miguel Miranda Quaresma A77049}
\institute{                                                                
University of Minho, Department of  Informatics, Braga, Portugal\\
e-mail: \{a78821,a77049\}@alunos.uminho.pt
}

\maketitle

\justify

\begin{abstract}
Este documento serve de apoio ao desenvolvimento de um processador Thesaurus explicando todas as decisões tomadas de modo a obter o mesmo. Inicialmente será explicado a estrutura dos ficheiros de entrada (extensão .mdic) bem como será mostrado as decisões tomadas de modo a chegar ao produto final. Por fim, será apresentado exemplos de utilização do mesmo.
\end{abstract}

\section{Introdução}
O presente projeto (Processador de Thesaurus) visa ser um sistema de processamento de dicionários Thesaurus que recorre a expressões regulares (ERs) para filtrar e transformar os mesmos, extraindo e tratando a informação mais relevante de forma eficiente. Para atingir este objetivo é usada a linguagem de filtragem e tratamento de dados \texttt{GAWK} visto ser uma \textbf{DSL}(Domain Specific Language) com foco em dados semi-estruturados.


\section{Desenvolvimento}

\section{Conclusão}

\end{document}
